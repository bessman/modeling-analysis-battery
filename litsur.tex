%%%%%%%%%%%%%%%%%%%%%%%%%%%%%%%%%%%%%%%%%
% Simple Sectioned Essay Template
% LaTeX Template
%
% This template has been downloaded from:
% http://www.latextemplates.com
%
% Note:
% The \lipsum[#] commands throughout this template generate dummy text
% to fill the template out. These commands should all be removed when 
% writing essay content.
%
%%%%%%%%%%%%%%%%%%%%%%%%%%%%%%%%%%%%%%%%%

%----------------------------------------------------------------------------------------
%	PACKAGES AND OTHER DOCUMENT CONFIGURATIONS
%----------------------------------------------------------------------------------------

\documentclass[12pt]{article} % Default font size is 12pt, it can be changed here

\usepackage{geometry} % Required to change the page size to A4
\geometry{a4paper} % Set the page size to be A4 as opposed to the default US Letter

\usepackage[T1]{fontenc} % Enable in text < > symbols
\usepackage{lmodern}

\usepackage{graphicx} % Required for including pictures

\usepackage{float} % Allows putting an [H] in \begin{figure} to specify the exact location of the figure
\usepackage{wrapfig} % Allows in-line images such as the example fish picture

\usepackage{fixltx2e} % Enables text mode sub- and superscripts with \textsubscript{} and\textsuperscript{}

\usepackage{lipsum} % Used for inserting dummy 'Lorem ipsum' text into the template

\linespread{1.2} % Line spacing

%\setlength\parindent{0pt} % Uncomment to remove all indentation from paragraphs

\graphicspath{{Pictures/}} % Specifies the directory where pictures are stored

\begin{document}

%----------------------------------------------------------------------------------------
%	TITLE PAGE
%----------------------------------------------------------------------------------------

\begin{titlepage}

\newcommand{\HRule}{\rule{\linewidth}{0.5mm}} % Defines a new command for the horizontal lines, change thickness here

\center % Center everything on the page

\textsc{\LARGE University Name}\\[1.5cm] % Name of your university/college
\textsc{\Large Major Heading}\\[0.5cm] % Major heading such as course name
\textsc{\large Minor Heading}\\[0.5cm] % Minor heading such as course title

\HRule \\[0.4cm]
{ \huge \bfseries Title}\\[0.4cm] % Title of your document
\HRule \\[1.5cm]

\begin{minipage}{0.4\textwidth}
\begin{flushleft} \large
\emph{Author:}\\
John \textsc{Smith} % Your name
\end{flushleft}
\end{minipage}
~
\begin{minipage}{0.4\textwidth}
\begin{flushright} \large
\emph{Supervisor:} \\
Dr. James \textsc{Smith} % Supervisor's Name
\end{flushright}
\end{minipage}\\[4cm]

{\large \today}\\[3cm] % Date, change the \today to a set date if you want to be precise

%\includegraphics{Logo}\\[1cm] % Include a department/university logo - this will require the graphicx package

\vfill % Fill the rest of the page with whitespace

\end{titlepage}

%----------------------------------------------------------------------------------------
%	TABLE OF CONTENTS
%----------------------------------------------------------------------------------------

\tableofcontents % Include a table of contents

\newpage % Begins the essay on a new page instead of on the same page as the table of contents 

%----------------------------------------------------------------------------------------
%	INTRODUCTION
%----------------------------------------------------------------------------------------

\section{Introduction} % Major section

Example citation %\cite{Figueredo:2009dg}.

%------------------------------------------------

\subsection{Subsection 1} % Sub-section

\lipsum[1] % Dummy text

%------------------------------------------------

\subsection{Subsection 2} % Sub-section

\lipsum[2] % Dummy text

%------------------------------------------------

\subsubsection{Subsubsection 1} % Sub-sub-section

\lipsum[3] % Dummy text

%------------------------------------------------

\subsubsection{Subsubsection 2} % Sub-sub-section

\lipsum[4] % Dummy text

%----------------------------------------------------------------------------------------
%	MAJOR SECTION 1
%----------------------------------------------------------------------------------------

\section{The lithium-ion battery} % Major section

%------------------------------------------------

\subsection{History} % Sub-section

Lithium-ion battery technology is a development of the older lithium metal battery.
As the name suggests, this earlier technology used metallic lithium in place of graphite for the negative electrode.
The lithium metal battery was first described in the mid 70s by M. S. Whittingham \cite{whittingham_electrical_1976}.
While this battery type has a very high energy density \cite{van_schalkwijk_chapter_2002}, the presence of metallic lithium in the cells is a serious safety hazard \cite{lisbona_review_2011} \cite{eom_life_2007}.
In the mid 80s, Akira Yoshino patented a cell design which used Li\textsuperscript{+}-ions as the primary charge carrier between to non-metallic electrodes \cite{yoshio_chapter_2009} \cite{yoshino_secondary_1987}.
Instead of lithium, the negative electrode used carbon as the active material.
Since the lithium-ion battery was first commercialized by Sony in 1991, the technology has grown to dominate the mobile appliances market \cite{yoshio_chapter_2009-1}.
It has also received considerable research effort, greatly improving energy density, power density, longevity, and cost efficiency.
Due to this, lithium-ion batteries are now poised to do to the electric vehicle market what it did to the mobile appliances market twenty years ago \cite{reddy_thomas_section_2011}.
However, many challenges remain before lithium-ion batteries are suitable for large-scale deployment in electrical vehicles.
Further improvements to cost efficiency, or price per kWh, and longevity are especially important [citation needed].

%------------------------------------------------

\subsection{Chemistry} % Sub-section

The basic operation of lithium-ion batteries involves reversible insertion and removal of Li\textsuperscript{+}-ions in metal oxides and graphite.
This process is called intercalation.
As a lithium-ion cell is charged, Li\textsuperscript{+}-ions deintercalate from the positive electrode active material and migrate through the electrolyte to the negative electrode where they are inserted into the active material.
The process is reversed during discharge \cite{reddy_thomas_section_2011-1}.

\subsubsection{The positive electrode} % Sub-sub-section

Typical positive electrode materials are lithium-cobalt oxides, lithium-nickel oxides, lithium manganese oxide spinels, and more recently lithium-iron phosphates \cite{kulova_new_2013}.
Aside from LiFePO\textsubscript{4}, these materials all operate at potentials positive of 3.5 V during the entire charge/discharge cycle, which is not always ideal from a safety perspective.
\\
Although LiCoO\textsubscript{2} is still the most common positive electrode material, LiFePO\textsubscript{4} is gaining traction due to its increased safety properties.
The specific energy of LiFePO\textsubscript{4} (160 mAh/g) is comparable to that of LiCoO\textsubscript{2} (155 mAh/g), but less than LiNi\textsubscript{0.8}Co\textsubscript{0.15}Al\textsubscript{0.05}O\textsubscript{2} (200 mAh/g).
As iron is considerably cheaper than cobalt, the raw material cost of LiFePO\textsubscript{4} is correspondingly lower than for LiCoO\textsubscript{2}.
However, this effect is offset by LiFePO\textsubscript{4} being more expensive to produce.
As LiFePO\textsubscript{4} contains both bivalent and trivalend iron, it must be prepared in inert atmosphere to prevent the bivalent iron from oxidizing.
LiFePO\textsubscript{4} particles must also be small (< 500 nm) to provide high enough current density.
\\
Another promising material is Li[Li\textsubscript{1/9}NI\textsubscript{1/3}Mn\textsubscript{5/9}]O\textsubscript{2}, which has a specific capacity of 275 mAh/g.
This material is not currently commercialized due to low rate capability \cite{reddy_thomas_section_2011-1}.
\\
Vanadium oxide based positive electrodes are also under development.
The above mentioned Co, Ni, Mn, and Fe based materials all have a maximum valency change of 1 when intercalating lithium; vanadium oxide (V\textsubscript{2}O\textsubscript{5}) can theoretically change valency by 3 units.
This allows for a higher specific capacity: 450 mAh/g has been achieved for thin-film electrodes.
However, the specific volume of this material changes drastically when this much lithium is inserted, which is a problem for lifetime \cite{kulova_new_2013}.
\\
Work is also ongoing to find electrode materials that operate at potentials higher than 4.5 V.
Although such high potentials are problematic from a safety perspective, they have the potential to offer increased energy density.
One such material is LiMn\textsubscript{1/3}Ni\textsubscript{1/3}Co\textsubscript{1/3}O\textsubscript{4}.
The primary obstacle for this class of materials is the need for electrolytes which are stable at these high potentials \cite{kulova_new_2013}.

%------------------------------------------------

\subsubsection{The negative electrode} % Sub-sub-section

Most commercial Li-ion batteries used graphite as the negative electrode active material, but other materials, foremost among them Li\textsubscript{3/4}Ti\textsubscript{5/4}O\textsubscript{4} has recently begun making headway in this market \cite{kulova_new_2013}.
Sony is using tin based negative electrodes in some of their commercial cells \cite{reddy_thomas_section_2011-2}.
The potential of lithium intercalation into graphite is only 0.1 V - 0.2 V positive of lithium's reduction potential (-3.045 V vs. SHE).
Combined with graphite's high specific capacity of about 360 mAh/g, this property makes graphite an excellent substitute for metallic lithium \cite{reddy_thomas_figure_2011} \cite{reddy_thomas_figure_2011-1}.
However, the specific volume of graphite can change by as much as 12\% during intercalation, which causes stress in the binder material \cite{kulova_new_2013}.
Additionally, graphite readily intercalates not only lithium ions, but also propylene carbonate (PC), which is frequently used as a solvent for the electrolyte.
If PC is intercalated the graphite sheets will separate, a process known as exfoliation \cite{reddy_thomas_section_2011-2}.
In light of these flaws, considerable effort is spent on developing alternative materials for the negative electrode.
\\
Many researchers have tried to use various metal oxides for this purpose.
The first such attempt was reported in 1997 \cite{idota_tin-based_1997}, using a tin-based oxide.
This technology seemed promising, boasting specific capacities as high as 600 mAh/g, but almost half of this is lost within the first 30 cycles.
Nevertheless, work on this material is still ongoing \cite{kulova_new_2013}.
Since then, a large body of work has been created, exploring a large number of metal oxides.
None has proven superior to graphite, with either the potential vs. lithium being too high or the specific capacity being too low.
One metal oxide of particular interest is lithium titanite (Li\textsubscript{3/4}Ti\textsubscript{5/4}O\textsubscript{4}).
Although inferior to graphite in terms of raw electrochemical performance, this material exhibits virtually zero change in volume during intercalation.
The electrochemical performance is also good enough to not be wholly discouraging, with a specific capacity of 175 mAh/g and an intercalation potential approximately 1 V above lithium \cite{kulova_new_2013} \cite{reddy_thomas_figure_2011}.
\\
Another candidate for the negative electrode material are metal and metal composites.
Most composites include tin due to the metal's ability to intercalate large amounts of lithium.
One such compound is Cu\textsubscript{6}Sn\textsubscript{5}, which Hu et al. used to achieve discharge a capacity of 370 mAh/g in the 0.1 V to 1.25 V region for 70 cycles \cite{hu_cyclic_2009}.
However, cyclic voltammetry of this material showed a large potential gap of up to 0.5 V between the intercalation and deintercalation reactions, which is bad news for the energy efficiency of the hypothetical resulting cell.
Overall, metal composite negative electrode materials also display poor cyclability and modest specific capacity at the potentials required of a lithium-ion battery negative electrode.
As such, these materials are not considered strong contenders for this application \cite{kulova_new_2013}.
\\
Another potential material is silicon.
Theoretically, 22 lithium atoms can be inserted per 5 silicon atoms, which corresponds to a specific reversible capacity of 2000 mAh/g.
Unfortunately this is accompanied by a large increase in specific volume for the silicon crystals.
A possible way to overcome this obstacle is the use of ultra thin (~100 nm) silicon electrodes, which have been shown to have good cycling stability with excellent specific capacity retention \cite{kulova_new_2013}.
\\
Linden's Handbook of Batteries describes both Si and Sn based negative electrodes as "almost commercially viable" \cite{reddy_thomas_section_2011-2}.

%------------------------------------------------

\subsubsection{The electrolyte} % Sub-sub-section

Since lithium-ion cells operate outside of the stability region of water, electrolytes must necessarily be nonaqueous.

\cite{reddy_thomas_section_2011-3}

Salts.

Solvents.

Additives.

\subsubsection{Binder materials} % Sub-sub-section

\begin{description} % Numbered list example

\item[First] \hfill \\
\lipsum[9] % Dummy text

\item[Second] \hfill \\
\lipsum[10] % Dummy text

\item[Third] \hfill \\
\lipsum[11] % Dummy text

\end{description} 

%------------------------------------------------

\subsection{Safety}

%------------------------------------------------

\subsection{Battery management systems}

%----------------------------------------------------------------------------------------
%	MAJOR SECTION X - TEMPLATE - UNCOMMENT AND FILL IN
%----------------------------------------------------------------------------------------

%\section{Content Section}

%\subsection{Subsection 1} % Sub-section

% Content

%------------------------------------------------

%\subsection{Subsection 2} % Sub-section

% Content

%----------------------------------------------------------------------------------------
%	CONCLUSION
%----------------------------------------------------------------------------------------

\section{Conclusion} % Major section

\lipsum[12-13]

%----------------------------------------------------------------------------------------
%	BIBLIOGRAPHY
%----------------------------------------------------------------------------------------

\bibliography{LiBat.bib}
\bibliographystyle{plain}

%----------------------------------------------------------------------------------------

\end{document}
