%%%%%%%%%%%%%%%%%%%%%%%%%%%%%%%%%%%%%%%%%
% Simple Sectioned Essay Template
% LaTeX Template
%
% This template has been downloaded from:
% http://www.latextemplates.com
%
% Note:
% The \lipsum[#] commands throughout this template generate dummy text
% to fill the template out. These commands should all be removed when 
% writing essay content.
%
%%%%%%%%%%%%%%%%%%%%%%%%%%%%%%%%%%%%%%%%%

%----------------------------------------------------------------------------------------
%	PACKAGES AND OTHER DOCUMENT CONFIGURATIONS
%----------------------------------------------------------------------------------------

\documentclass[12pt]{article} % Default font size is 12pt, it can be changed here

\usepackage{geometry} % Required to change the page size to A4
\geometry{a4paper} % Set the page size to be A4 as opposed to the default US Letter

\usepackage[T1]{fontenc} % Enable < > symbols in verbatim
\usepackage{fancyvrb} % Used to define commands for other symbols
\usepackage{fontspec}
\usepackage{lmodern}

\usepackage{gensymb} % \degree = ° 

\usepackage{graphicx} % Required for including pictures

\usepackage{float} % Allows putting an [H] in \begin{figure} to specify the exact location of the figure
\usepackage{wrapfig} % Allows in-line images such as the example fish picture

\usepackage{fixltx2e} % Enables text mode sub- and superscripts with \textsubscript{} and\textsuperscript{}

\usepackage{lipsum} % Used for inserting dummy 'Lorem ipsum' text into the template

\linespread{1.2} % Line spacing

%\setlength\parindent{0pt} % Uncomment to remove all indentation from paragraphs

\graphicspath{{Pictures/}} % Specifies the directory where pictures are stored

\begin{document}

%----------------------------------------------------------------------------------------
%	COMMAND DEFINITIONS
%----------------------------------------------------------------------------------------

\newcommand{\mytilde}{$\sim$}
\begin{Verbatim}[commandchars=\\\{\}]
sig symm : (Board, [(Int, Int)]) \mytilde> Bool
\end{Verbatim}

%----------------------------------------------------------------------------------------
%	TITLE PAGE
%----------------------------------------------------------------------------------------

\begin{titlepage}

\newcommand{\HRule}{\rule{\linewidth}{0.5mm}} % Defines a new command for the horizontal lines, change thickness here

\center % Center everything on the page

\textsc{\LARGE University Name}\\[1.5cm] % Name of your university/college
\textsc{\Large Major Heading}\\[0.5cm] % Major heading such as course name
\textsc{\large Minor Heading}\\[0.5cm] % Minor heading such as course title

\HRule \\[0.4cm]
{ \huge \bfseries Title}\\[0.4cm] % Title of your document
\HRule \\[1.5cm]

\begin{minipage}{0.4\textwidth}
\begin{flushleft} \large
\emph{Author:}\\
John \textsc{Smith} % Your name
\end{flushleft}
\end{minipage}
~
\begin{minipage}{0.4\textwidth}
\begin{flushright} \large
\emph{Supervisor:} \\
Dr. James \textsc{Smith} % Supervisor's Name
\end{flushright}
\end{minipage}\\[4cm]

{\large \today}\\[3cm] % Date, change the \today to a set date if you want to be precise

%\includegraphics{Logo}\\[1cm] % Include a department/university logo - this will require the graphicx package

\vfill % Fill the rest of the page with whitespace

\end{titlepage}

%----------------------------------------------------------------------------------------
%	TABLE OF CONTENTS
%----------------------------------------------------------------------------------------

\tableofcontents % Include a table of contents

\newpage % Begins the essay on a new page instead of on the same page as the table of contents 

%----------------------------------------------------------------------------------------
%	INTRODUCTION
%----------------------------------------------------------------------------------------

\section{Introduction} % Major section

Example citation %\cite{Figueredo:2009dg}.

%------------------------------------------------

\subsection{Subsection 1} % Sub-section

\lipsum[1] % Dummy text

%------------------------------------------------

\subsection{Subsection 2} % Sub-section

\lipsum[2] % Dummy text

%------------------------------------------------

\subsubsection{Subsubsection 1} % Sub-sub-section

\lipsum[3] % Dummy text

%------------------------------------------------

\subsubsection{Subsubsection 2} % Sub-sub-section

\lipsum[4] % Dummy text

%----------------------------------------------------------------------------------------
%	MAJOR SECTION 1
%----------------------------------------------------------------------------------------
\section{The lithium-ion battery} % Major section

%------------------------------------------------

\subsection{History} % Sub-section

Lithium-ion battery technology is a development of the older lithium metal battery.
As the name suggests, this earlier technology used metallic lithium in place of graphite for the negative electrode.
The lithium metal battery was first described in the mid 70s by M. S. Whittingham \cite{whittingham_electrical_1976}.
While this battery type has a very high energy density \cite{van_schalkwijk_chapter_2002}, the presence of metallic lithium in the cells is a serious safety hazard \cite{lisbona_review_2011} \cite{eom_life_2007}.
In the mid 80s, Akira Yoshino patented a cell design which used Li\textsuperscript{+}-ions as the primary charge carrier between to non-metallic electrodes \cite{yoshio_chapter_2009} \cite{yoshino_secondary_1987}.
Instead of lithium, the negative electrode used carbon as the active material.
Since the lithium-ion battery was first commercialized by Sony in 1991, the technology has grown to dominate the mobile appliances market \cite{yoshio_chapter_2009-1}.
It has also received considerable research effort, greatly improving energy density, power density, longevity, and cost efficiency.
Due to this, lithium-ion batteries are now poised to do to the electric vehicle market what it did to the mobile appliances market twenty years ago \cite{reddy_thomas_section_2011}.
However, many challenges remain before lithium-ion batteries are suitable for large-scale deployment in electrical vehicles.
Further improvements to cost efficiency, or price per kWh, and longevity are especially important.

%------------------------------------------------

\subsection{Chemistry} % Sub-section

The basic operation of lithium-ion batteries involves reversible insertion and removal of Li\textsuperscript{+}-ions in metal oxides and graphite.
This process is called intercalation.
As a lithium-ion cell is charged, Li\textsuperscript{+}-ions deintercalate from the positive electrode active material and migrate through the electrolyte to the negative electrode where they are inserted into the active material.
The process is reversed during discharge \cite{reddy_thomas_section_2011-1}.

\subsubsection{The positive electrode} % Sub-sub-section

Typical positive electrode materials are lithium-cobalt oxides, lithium-nickel oxides, lithium-manganese oxide spinels, and more recently lithium-iron phosphates \cite{kulova_new_2013}.
The working principle of all these materials is the same; reversible intercalation of lithium in the crystal structure of the host material.
\\
As far as the working principle is concered, all positive electrode materials are equal.
Where they differ is in their specific capacity, potential range, cost, and stability.

\begin{description} % Numbered list example

\item[Lithium cobalt oxide] \hfill \\
LiCoO\textsubscript{2} was one of the materials mentioned by Yoshino in the original patent, and is still the most common positive electrode material used today \cite{yoshino_secondary_1987}.
Its specific capacity is 155 mAh/g and its potential profile is flat around 4 V.

\item[Lithium iron phosphate] \hfill \\
LiFePO\textsubscript{4} is gaining traction due to its decreased cost and increased safety properties compared to LiCoO\textsubscript{2}.
As iron is considerably cheaper than cobalt, the raw material cost of LiFePO\textsubscript{4} is correspondingly lower than for LiCoO\textsubscript{2}.
However, this effect is offset by LiFePO\textsubscript{4} being more expensive to produce.
As LiFePO\textsubscript{4} contains bivalent iron, it must be prepared in inert atmosphere to prevent the iron from oxidizing.
LiFePO\textsubscript{4} particles must also be small (< 500 nm) to provide high enough current density.
Its specific capacity is 160 mAh/g and its potential profile is flat at 3.5 V.
Although its low potential decreases its specific energy, it simultaneously improves the safety of the material.

\item[{Li[Li\textsubscript{1/9}NI\textsubscript{1/3}Mn\textsubscript{5/9}]O\textsubscript{2}}] \hfill \\
This material has a specific capacity of 275 mAh/g and a midpoint potential of 3.8 V.
It is not currently commercialized due to low rate capability \cite{reddy_thomas_section_2011-1}.

\item[Lithium vanadium oxide] \hfill \\
Co, Ni, Mn, and Fe based materials all have a maximum valency change of 1 when intercalating lithium; vanadium oxide (V\textsubscript{2}O\textsubscript{5}) can theoretically change valency by 3 units.
This allows for a higher specific capacity: 450 mAh/g has been achieved for thin-film electrodes.
However, the specific volume of this material changes drastically when this much lithium is inserted, which is a problem for lifetime \cite{kulova_new_2013}.

\item[LiMn\textsubscript{1/3}Ni\textsubscript{1/3}Co\textsubscript{1/3}O\textsubscript{2}] \hfill \\
Work is also ongoing to find electrode materials that operate at potentials higher than 4.5 V.
Although such high potentials are problematic from a safety perspective, they have the potential to offer increased energy density.
One such material is LiMn\textsubscript{1/3}Ni\textsubscript{1/3}Co\textsubscript{1/3}O\textsubscript{2}.
The primary obstacle for this class of materials is the need for electrolytes which are stable at these high potentials \cite{kulova_new_2013}.

\end{description} 

%------------------------------------------------

\subsubsection{The negative electrode} % Sub-sub-section

Unlike the lithium metal battery, lithium-ion batteries use an intercalating material not only at the positive electrode, but also at the negative.
Most commercial Li-ion batteries used graphite as the negative electrode active material, but other materials, foremost among them Li\textsubscript{3/4}Ti\textsubscript{5/4}O\textsubscript{4} has recently begun making headway in this market \cite{kulova_new_2013}.
There are also some materials which form alloys with lithium instead of intercalation.

\begin{description} % Numbered list example

\item[Graphite] \hfill \\
The potential of lithium intercalation into graphite is only 0.1 V - 0.2 V positive of lithium's reduction potential (-3.045 V vs. SHE).
Combined with graphite's high specific capacity of about 360 mAh/g, this property makes graphite an excellent substitute for metallic lithium \cite{reddy_thomas_figure_2011} \cite{reddy_thomas_figure_2011-1}.
However, the specific volume of graphite can change by as much as 12\% during intercalation, which causes stress in the binder material \cite{kulova_new_2013}.
Additionally, graphite readily intercalates not only lithium ions, but also propylene carbonate (PC), which is frequently used as a solvent for the electrolyte.
If PC is intercalated the graphite sheets will separate, a process known as exfoliation \cite{reddy_thomas_section_2011-2}.
In light of these flaws, considerable effort is spent on developing alternative materials for the negative electrode.

\item[Lithium titanate] \hfill \\
One metal oxide of particular interest is lithium titanate (Li\textsubscript{3/4}Ti\textsubscript{5/4}O\textsubscript{4}).
Its specific capacity is 175 mAh/g and its intercalation potential is approximately 1.5 V above lithium \cite{kulova_new_2013} \cite{reddy_thomas_figure_2011}.
It therefore cannot provide as much charge or as much energy as a comparable amount of graphite.
Although inferior to graphite in terms of specific capacity and potential, this material exhibits virtually zero change in volume during intercalation.

\item[Lithium tin oxide] \hfill \\

An earlier attempt at using metal oxides was reported in 1997 \cite{idota_tin-based_1997}, using a tin-based oxide.
This technology seemed promising, boasting specific capacities as high as 600 mAh/g, but almost half of this is lost within the first 30 cycles.
Nevertheless, work on this material is still ongoing \cite{kulova_new_2013}.
In 2005 Sony began using tin based negative electrodes in their 2.2 Ah Nexelion battery \cite{sony_sony_????}.
The Nexelion was updated to 3.5 Ah in 2011 \cite{sony_sony_????-1}.

\item[Lithium alloys] \hfill \\
Some negative electrode materials deviate from the intercalation paradigm, operating instead by forming lithium alloys.
One such compound is Cu\textsubscript{6}Sn\textsubscript{5}, which Hu et al. used to achieve discharge a capacity of 370 mAh/g in the 0.1 V to 1.25 V region for 70 cycles \cite{hu_cyclic_2009}.
However, cyclic voltammetry of this material showed a large potential gap of up to 0.5 V between the charge and discharge reactions, which is bad news for the energy efficiency of the hypothetical resulting cell.
Overall, metal alloy negative electrode materials also display poor cyclability and modest specific capacity at the potentials required of a lithium-ion battery negative electrode.
As such, these materials are not considered strong contenders for this application \cite{kulova_new_2013}.

\item[Silicon] \hfill \\

Another potential material is silicon.
Theoretically, 22 lithium atoms can be inserted per 5 silicon atoms, which corresponds to a specific reversible capacity of 2000 mAh/g.
Unfortunately this is accompanied by a large increase in specific volume for the silicon crystals.
A possible way to overcome this obstacle is the use of ultra thin (\mytilde{}100 nm) silicon electrodes, which have been shown to have good cycling stability with excellent specific capacity retention \cite{kulova_new_2013}.
 
\end{description} 

Linden's Handbook of Batteries describes both Si and Sn based negative electrodes as "almost commercially viable" \cite{reddy_thomas_section_2011-2}.

%------------------------------------------------

\subsubsection{The electrolyte} % Sub-sub-section

Since lithium-ion cells operate outside of the stability region of water, the electrolyte must necessarily be nonaqueous.
It typically consists of an organic solvent and a lithium salt, although solid electrolytes also exist.
Solid electrolytes are also called gel electrolytes or polymer electrolytes \cite{reddy_thomas_section_2011-3}.
\\
As of today, lithium phosphorus hexafluoride, LiPF\textsubscript{6}, is the most used salt for lithium-ion batteries due to its "high ionic conductivity (10 mS/cm), high lithium-ion transference number (\mytilde{}0.35), and acceptable safety properties" \cite{reddy_thomas_section_2011-3}.
Additionally, it has the advantageous ability to passivate aluminium at very high potentials (over 4.5 V vs lithium redox), which is frequently used as a current collector on the positive electrode \cite{niedzicki_new_2011}.
However, this salt has many negative properties, which has spurred research into finding a suitable replacement.
For example, it reacts with water it forms HF, which corrodes the positive electrode active material.
It also has a fairly narrow stability range, and is especially sensitive to high temperatures.
At temperatures above 80\degree C it reacts with ethylene carbonate (a common solvent) to form extremely toxic flouroethanol \cite{hammami_lithium-ion_2003}.
Other salts which are sometimes used are LiBF\textsubscript{6}, LiB(C\textsubscript{2}O\textsubscript{4})\textsubscript{2} (LiBOB), and LiN(CF\textsubscript{3}SO\textsubscript{2})\textsubscript{2} (LiNTF\textsubscript{2}).
These salts have various advantages over LiPF\textsubscript{6}, such as being more stable when exposed to water, and having improved high temperature performance \cite{reddy_thomas_section_2011-3}.
Niedzicki et al. recently presented two new salts for high voltage cells operating beyond 4 V.
These salts, lithium(2-trifluoromethyl-4,5-dicyano-imidazolate) and lithium(2-pentafluoroethyl-4,5-dicyano-imidazolate) have a cyclic structure and are also able to passivate aluminium.
Their conductivity is lower than that of LiPF\textsubscript{6}, but they are more thermally stable and release less HF during combustion \cite{niedzicki_new_2011}.
\\
The salt is dissolved in a mixture of organic liquids, or in a high-molecular-weight polymer in the case of solid electrolytes.
Different compounds such as carbonates, ethers, and acetates have properties both positive and negative to the performance of the cell, and by finding the correct mixture one can minimize each compound's negative effects while maximizing its positive effects.
For example, ethylene carbonate (EC) improves the irreversible capacity of the cell, but is solid in room temperature.
By combining it with other solvents with lower melting points this desirable property can benefit batteries which will be used at ambient temperatures.
Commercial cells use three to five organic compounds in the solvent, plus additives.
The conductivity of the electrolyte depends primarily on the solvent mixture, the salt concentration, and the temperature.
For LiPF\textsubscript{6} in a 1:1 mixture of EC and another solvent, the order of magnitude is 1 - 10 mS/cm.
The conductivity increases with salt concentration and electrolyte temperature \cite{reddy_thomas_section_2011-3}.
In 2011, Kamaya et al. presented a solid electrolyte, Li\textsubscript{10}GEP\textsubscript{2}S\textsubscript{12}, which has an ionic conductivity of 12 mS/cm at room temperature.
This is conductivity slightly higher than what is typically achievable with LiPF\textsubscript{6}-based liquid electrolytes, and Kamaya's solid electrolyte is also safer \cite{kamaya_lithium_2011}.
Ionic liquids have also been considered as electrolytes, but since they are neither stable at the negative electrode potential nor are able to passivate it, attempts to use them have so far been unsuccesfull \cite{blomgren_liquid_2003}.
\\
Another desirable property of the electrolyte is the ability to form a stable passivating layer on the negative electrode.
Since the negative electrode operates at just a few mV above lithium's redox potential, no electrolyte is stable in its vicinity.
It is therefore important that the electolyte's reaction at the negative electrode form a layer that is impenetrable to the electrolyte itself, yet allows lithium-ions to move freely across it.
The layer must also be stable over time and over the entire potential span of the negative electrode.
Such a layer is called a solid electrolyte interphase (SEI) layer \cite{reddy_thomas_section_2011-3}.
\\
Different chemicals are often added to the electrolyte to improve its properties.
These chemicals are not themselves charge-carryers or solvents, and are simply called additives.
Some of the properties which can be improved by additives are:

\begin{description} % Numbered list example

\item[Flammability] \hfill \\
Or rather, the lack thereof.
Organic solvents are highly flammable, and if a thermal runaway reaction occurs it can be both violent and difficult to stop.
By adding a compound which prevents the chain reaction from proceeding once the external heat source has been removed damage to surrounding components can be minimized, although the battery itself will of course be unsalvagable \cite{blomgren_liquid_2003}.

\item[SEI formation and stability] \hfill \\
Without the presence of SEI forming additives, the SEI must be formed by using up other electrolyte components.
Since the SEI formation is irreversible, this causes a capacity loss.
If instead the SEI can be formed by non-charge-carrying species, this capacity loss can be reduced.
Additionally, by improving the stability of the SEI further capacity loss can be avoided \cite{reddy_thomas_section_2011-4}.

\item[Overcharge protection] \hfill \\
Normally, if a lithium ion battery is charged beyond its nominal voltage it will degrade rapidly as the electrolyte starts to decompose into highly reactive species such as HF, which in turn will damage other components in the cell.
This is not a problem in certain other battery chemistries, such as lead-acid.
There, the aqueous electrolyte will undergo electrolysis and form hydrogen and water, which will soon recombine into water after the overcharging stops.
Overcharge protection in lithium-ion batteries work by a similar principle; a species is added which is more reactive than the electrolyte at high potentials and reacts at the positive electrode to form a cation which is reduced back to its original form at the negative electrode.
This is called a redox shuttle \cite{blomgren_liquid_2003} \cite{reddy_thomas_section_2011-4}.

\end{description} 

\subsubsection{Separator materials} % Sub-sub-section

In cells which use a liquid electrolyte, an electrical isolator must be present between the electrodes to prevent short-circuits.
This isolator must be permeable to the electrolyte and is termed the separator.
In lithium-ion cells the separator is typically 16-40 µm in width and is made of the polyalkenes polyethylene and polypropylene.
These materials provide good mechanical properties, including the ability to be tightly wound in one direction without yielding or shrinking in width, puncture resistance, and high wettability by the electrolyte.
The separator is microporous with pore sizes of 0.03 µm to 0.1 µm and between 30\% to 50\% porosity\cite{reddy_thomas_section_2011-5}.
\\
The separator also provides additional safety functionality, since polyethylene melts at 135\degree C and polypropylene at 155\degree C.
At these temperatures the pores in the separator will close, thereby preventing lithium ions from migrating between the electrodes and effectively halting the cell reactions from proceeding.
However, if the separator melts and flows, the electrical isolation may be compromised.
It is therefore desirable to use an additional, heat-resistant material in the separator formulation.
Some manufacturers have used refractory metal oxides such as Al \textsubscript{2}O\textsubscript{3} for this purpose \cite{reddy_thomas_section_2011-5}.

%------------------------------------------------

\section{Capacitor chemistry}

This project is about the interaction between chemistry and electrical machines, and can be said to roughly involve three critical components: The battery, the AC/DC converter, and the DC-link capacitor, which we hope to eliminate.
The battery lies firmly within the realm of chemistry and the AC/DC converter within the realm of electrical engineering.
The capacitor, on the other hand, is a little of both.
Although traditionally viewed as a purely electrical component, several common capacitor types operate by partially or wholly chemical mechanisms (for example, the electrolytical and electrochemical capacitor families).

\subsection{Ceramic capacitors}

The most widely used capacitor type with 90\% marketshare in terms of units sold and 40\% in terms of unit value, the ceramic capacitor is not particularly interesting from a chemical perspective.
It consists of two metal plates with a dielectric material between them.
Although there is obviously some chemistry involved in the formulation of the dielectric, once it's in place there is no further chemistry involved in the energy storage mechanism itself.
Since ceramic capacitors contains no liquid, they can withstand high temperatures and voltages, up to hundreds of degrees Celsius and thousands of Volts \cite{pan_brief_2010}.

\subsection{Electrolytical capacitors}

As the name suggests, the electrolytical capacitor contains an electrolyte, usually an ionically condutive liquid.
However, the energy storage mechanism is not electrochemical in nature, but rather purely electrostatic like most capacitor types.
It consists of two sheets of metal foil, between which is a sheet of paper soaked in electrolyte.
The metal is typically aluminium due to its ability to form a thin, passivating oxide layer in the presence of oxygen.
This oxide layer is the dielectric of the electrolytic capacitor.
Therefore the electrostatic charge is actually stored between one of aluminium sheets and the electrolyte itself, and not between the two metal sheets.
In effect, the electrolyte is an electrode, and the other aluminium sheet which is not involved in the charge storage is little more than a current collector.
The aluminium foil on which the dielectric oxide layer is located is always the positive electrode, because if the polarity of the capacitor were reversed the oxide layer would dissolve and a short-circuit would form, destroying the capacitor.
\\
The performance of an aluminium based electrolytic capacitor can be influenced in several ways.
Recalling that the capacitance is inversely proportional to the thickness of the dielectric material, one way to increase the capacity of an electrolytic capacitor is to make the aluminium oxide layer as thin as possible.
However, with a thicker oxide layer the capacitor can withstand higher voltages, at the cost of capacitance.
The capacitance is also directly proportional to the surface area of the electrodes, which may be increased by roughening the aluminium foils in a process called "etching" \cite{cdm_cornell_dubilier_aeappguide.pdf_????}.
\\
Figure \ref{fig:electrolytic} shows an illustrated cross-section of an electrolytic capacitor.
\begin{figure}[H] % Example image
\center{\includegraphics[width=0.5\linewidth]{Bilder/Elko-Prinzipschnittbild-english.png}}
\caption{Schematic overview of an electrolytic capacitor.}
\label{fig:electrolytic}
\end{figure}

\subsection{Electrochemical capacitors}

Unlike other capacitor types, the energy storage mechanism in electrochemical capacitors is only partially electrostatic in nature.
Charge is also stored faradaically via chemical reactions, a phenomenon which is called "pseudocapacitance".
The electrostatic energy storage is due to charge separation across the Helmholz layer, which is a plane of ions at each electrode, where the ions are of opposite charge compared to the electrodes.
The distance between the capacitor "plates" is very small, being roughly equal to half the diameter of the ions, meaning the order of magitude is measured in Ångströms.
This results in high specific capacitances of 15 - 30 µF/cm\textsuperscript{2}.
The active material in electrochemical capacitors is usually active carbon, which has a surface area in the order of magnitude of 1000 m\textsuperscript{2}/g.
Combined, these numbers yield capacitances measured in hundreds of Farads per gram of active material, which is far beyond what any other type of capacitor can achieve.
\\
The other storage mechanism, pseudocapacitance, is so called because it is in fact not electrostatic in nature, and thus not a true capacitance.
It involves charge transfer reations on the electrode surfaces, which are voltage dependent.
Because of this the capacitance of electrochemical capacitors is not constant with regard to the voltage.
\\
Electrochemical capacitors can be manufactured to make more of less use of one or the other energy storage mechanism, but both types will always be present to some degree \cite{reddy_thomas_section_2011-5}.

%----------------------------------------------------------------------------------------
%	MAJOR SECTION X - TEMPLATE - UNCOMMENT AND FILL IN
%----------------------------------------------------------------------------------------

%\section{Content Section}

%\subsection{Subsection 1} % Sub-section

% Content

%------------------------------------------------

%\subsection{Subsection 2} % Sub-section

% Content

%----------------------------------------------------------------------------------------
%	CONCLUSION
%----------------------------------------------------------------------------------------

\section{Conclusion} % Major section

\lipsum[12-13]

%----------------------------------------------------------------------------------------
%	BIBLIOGRAPHY
%----------------------------------------------------------------------------------------

\bibliography{LiBat.bib}
\bibliographystyle{plain}

%----------------------------------------------------------------------------------------

\end{document}
